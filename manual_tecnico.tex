\documentclass[12pt,a4paper]{report}
\usepackage[utf8]{inputenc}
\usepackage[spanish]{babel}
\usepackage{geometry}
\usepackage{graphicx}
\usepackage{fancyhdr}
\usepackage{titlesec}
\usepackage{listings}
\usepackage{xcolor}
\usepackage{hyperref}
\usepackage{amsmath}
\usepackage{amssymb}
\usepackage{enumitem}
\usepackage{booktabs}
\usepackage{array}
\usepackage{longtable}

% Configuración de página
\geometry{margin=2.5cm}
\pagestyle{fancy}
\fancyhf{}
\fancyhead[L]{\leftmark}
\fancyhead[R]{\thepage}
\renewcommand{\headrulewidth}{0.4pt}

% Configuración de código
\definecolor{codegreen}{rgb}{0,0.6,0}
\definecolor{codegray}{rgb}{0.5,0.5,0.5}
\definecolor{codepurple}{rgb}{0.58,0,0.82}
\definecolor{backcolour}{rgb}{0.95,0.95,0.92}

\lstdefinestyle{mystyle}{
    backgroundcolor=\color{backcolour},   
    commentstyle=\color{codegreen},
    keywordstyle=\color{magenta},
    numberstyle=\tiny\color{codegray},
    stringstyle=\color{codepurple},
    basicstyle=\ttfamily\footnotesize,
    breakatwhitespace=false,         
    breaklines=true,                 
    captionpos=b,                    
    keepspaces=true,                 
    numbers=left,                    
    numbersep=5pt,                  
    showspaces=false,                
    showstringspaces=false,
    showtabs=false,                  
    tabsize=2
}

\lstset{style=mystyle}

% Configuración de títulos
\titleformat{\chapter}[display]
  {\normalfont\huge\bfseries}{\chaptertitlename\ \thechapter}{20pt}{\Huge}
\titlespacing*{\chapter}{0pt}{50pt}{40pt}

% Configuración de hipervínculos
\hypersetup{
    colorlinks=true,
    linkcolor=blue,
    filecolor=magenta,      
    urlcolor=cyan,
    pdftitle={Manual Técnico - Sistema de Generación de Exámenes con IA},
    pdfauthor={Equipo de Desarrollo},
}

\begin{document}

% Página de título
\begin{titlepage}
    \centering
    \vspace*{2cm}
    
    {\Huge\bfseries Manual Técnico}\\[0.5cm]
    {\Large Sistema de Generación de Exámenes con IA}\\[2cm]
    
    % \includegraphics[width=0.3\textwidth]{logo.png}\\[1cm] % Opcional: agregar logo
    
    {\large\itshape Equipo de Desarrollo}\\[3cm]
    
    {\large Versión 1.0}\\[0.5cm]
    {\large Junio 2025}
    
    \vfill
\end{titlepage}

% Tabla de contenidos
\tableofcontents
\newpage

% Lista de figuras (opcional)
\listoffigures
\newpage

% Lista de tablas (opcional)
\listoftables
\newpage

\chapter{Introducción}

\section{Propósito del Sistema}
El Sistema de Generación de Exámenes con IA es una plataforma web diseñada para crear automáticamente exámenes personalizados utilizando inteligencia artificial. El sistema permite a los usuarios subir documentos, generar preguntas basadas en el contenido, realizar exámenes con temporizador y recibir retroalimentación detallada.

\section{Alcance}
\begin{itemize}
    \item Generación automática de exámenes a partir de documentos
    \item Sistema de autenticación de usuarios
    \item Interfaz web responsive
    \item Integración con Google Gemini AI
    \item Almacenamiento en Supabase
    \item Retroalimentación personalizada
\end{itemize}

\section{Audiencia}
Este manual está dirigido a desarrolladores, administradores de sistema y personal técnico responsable del mantenimiento y desarrollo del sistema.

\chapter{Arquitectura del Sistema}

\section{Arquitectura General}
El sistema sigue una arquitectura de tres capas:

\begin{figure}[h]
\centering
\begin{verbatim}
+-----------------+    +-----------------+    +-----------------+
|   Frontend      |----|    Backend      |----|   Supabase      |
|   (React)       |    |   (Express)     |    |   (Database)    |
+-----------------+    +-----------------+    +-----------------+
                               |
                       +-----------------+
                       |  Google Gemini  |
                       |      AI         |
                       +-----------------+
\end{verbatim}
\caption{Arquitectura del Sistema}
\end{figure}

\section{Patrón Arquitectónico}
\begin{itemize}
    \item \textbf{Frontend}: SPA (Single Page Application) con React
    \item \textbf{Backend}: API RESTful con Express.js
    \item \textbf{Base de Datos}: PostgreSQL (Supabase)
    \item \textbf{Servicios Externos}: Google Gemini AI para generación de contenido
\end{itemize}

\chapter{Tecnologías Utilizadas}

\section{Frontend}
\begin{table}[h]
\centering
\begin{tabular}{ll}
\toprule
\textbf{Tecnología} & \textbf{Versión/Descripción} \\
\midrule
React & 19.0.0 - Framework principal \\
TypeScript & Tipado estático \\
Vite & Build tool y dev server \\
Tailwind CSS & Framework de estilos \\
React Router & Enrutamiento \\
Motion & Animaciones \\
SweetAlert2 & Alertas y modales \\
React Markdown & Renderizado de markdown \\
KaTeX & Renderizado de fórmulas matemáticas \\
\bottomrule
\end{tabular}
\caption{Tecnologías del Frontend}
\end{table}

\section{Backend}
\begin{table}[h]
\centering
\begin{tabular}{ll}
\toprule
\textbf{Tecnología} & \textbf{Descripción} \\
\midrule
Node.js & Runtime de JavaScript \\
Express.js & Framework web \\
Google GenAI & Integración con Gemini AI \\
Supabase & Base de datos y autenticación \\
Multer & Manejo de archivos \\
CORS & Cross-Origin Resource Sharing \\
dotenv & Variables de entorno \\
\bottomrule
\end{tabular}
\caption{Tecnologías del Backend}
\end{table}

\section{Base de Datos}
\begin{itemize}
    \item \textbf{Supabase (PostgreSQL)}: Base de datos principal
    \item \textbf{Autenticación}: Sistema de usuarios integrado
\end{itemize}

\chapter{Instalación y Configuración}

\section{Requisitos Previos}
\begin{itemize}
    \item Node.js 18+
    \item npm o yarn
    \item Cuenta de Supabase
    \item API Key de Google Gemini
\end{itemize}

\section{Instalación Frontend}
\begin{lstlisting}[language=bash]
cd frontend
npm install
\end{lstlisting}

\section{Instalación Backend}
\begin{lstlisting}[language=bash]
cd backend
npm install
\end{lstlisting}

\section{Variables de Entorno}

\subsection{Frontend (.env)}
\begin{lstlisting}[language=bash]
VITE_SUPABASE_URL=tu_supabase_url
VITE_SUPABASE_ANON_KEY=tu_supabase_anon_key
VITE_BACKEND_URL=http://localhost:3001
\end{lstlisting}

\subsection{Backend (.env)}
\begin{lstlisting}[language=bash]
GEMINI_API_KEY=tu_gemini_api_key
SUPABASE_URL=tu_supabase_url
SUPABASE_SERVICE_ROLE_KEY=tu_service_role_key
PORT=3001
\end{lstlisting}

\chapter{Estructura del Proyecto}

\section{Frontend}
\begin{lstlisting}
frontend/
+-- src/
    +-- API/
        +-- Gemini.tsx
    +-- components/
        +-- Main/
        +-- Examen/
        +-- ...
    +-- context/
        +-- AuthContext.tsx
    +-- pages/
        +-- Login.tsx
        +-- Examenes.tsx
        +-- ...
    +-- routers/
        +-- routes.tsx
    +-- ...
+-- public/
+-- package.json
+-- vite.config.ts
\end{lstlisting}

\section{Backend}
\begin{lstlisting}
backend/
+-- src/
    +-- index.js          # Servidor principal
    +-- reqAuthMiddleware.js
    +-- reqSupabase.js
    +-- local.js
    +-- analize.js
    +-- ...
+-- package.json
+-- supabase.config.js
\end{lstlisting}

\chapter{Componentes Principales}

\section{Frontend Components}

\subsection{AuthContext}
\begin{itemize}
    \item \textbf{Propósito}: Manejo global del estado de autenticación
    \item \textbf{Ubicación}: \texttt{src/context/AuthContext.tsx}
    \item \textbf{Funcionalidades}:
    \begin{itemize}
        \item Login/logout de usuarios
        \item Verificación de estado de sesión
        \item Protección de rutas
    \end{itemize}
\end{itemize}

\subsection{ExamenPage}
\begin{itemize}
    \item \textbf{Propósito}: Interfaz principal para realizar exámenes
    \item \textbf{Ubicación}: \texttt{src/Examen/ExamenPage.tsx}
    \item \textbf{Funcionalidades}:
    \begin{itemize}
        \item Renderizado de preguntas
        \item Timer de examen
        \item Selección de respuestas
        \item Envío de resultados
    \end{itemize}
\end{itemize}

\subsection{Navbar}
\begin{itemize}
    \item \textbf{Propósito}: Navegación principal
    \item \textbf{Ubicación}: \texttt{src/components/Navbar.tsx}
    \item \textbf{Funcionalidades}:
    \begin{itemize}
        \item Enlaces de navegación
        \item Estado de autenticación
        \item Menú responsivo
    \end{itemize}
\end{itemize}

\section{Backend Modules}

\subsection{Authentication Middleware}
\begin{itemize}
    \item \textbf{Propósito}: Verificación de tokens de usuario
    \item \textbf{Ubicación}: \texttt{src/reqAuthMiddleware.js}
    \item \textbf{Función}: \texttt{getUserFromRequest()}
\end{itemize}

\subsection{Supabase Integration}
\begin{itemize}
    \item \textbf{Propósito}: Operaciones de base de datos
    \item \textbf{Ubicación}: \texttt{src/reqSupabase.js}
    \item \textbf{Funciones principales}:
    \begin{itemize}
        \item \texttt{CreateAuthExamUser()}
        \item \texttt{CreateAuthFeedback()}
        \item \texttt{verifyAuthExamUser()}
    \end{itemize}
\end{itemize}

\chapter{API Endpoints}

\section{Generación de Exámenes}

\subsection{POST /api/upload\_files}
\begin{itemize}
    \item \textbf{Descripción}: Genera examen a partir de archivos subidos
    \item \textbf{Autenticación}: Requerida
    \item \textbf{Body}:
\end{itemize}

\begin{lstlisting}
{
  "prompt": "string",
  "tiempo_limite_segundos": "number"
}
\end{lstlisting}

\begin{itemize}
    \item \textbf{Archivos}: Multipart/form-data
    \item \textbf{Respuesta}: Objeto examen creado
\end{itemize}

\subsection{POST /api/generate-content}
\begin{itemize}
    \item \textbf{Descripción}: Genera examen a partir de prompt de texto
    \item \textbf{Autenticación}: Requerida
    \item \textbf{Body}:
\end{itemize}

\begin{lstlisting}
{
  "prompt": "string",
  "dificultad": "easy|medium|hard",
  "tiempo_limite_segundos": "number"
}
\end{lstlisting}

\subsection{POST /api/generate-content-based-on-history}
\begin{itemize}
    \item \textbf{Descripción}: Genera examen basado en historial de usuario
    \item \textbf{Autenticación}: Requerida
    \item \textbf{Body}:
\end{itemize}

\begin{lstlisting}
{
  "exams_id": ["array_of_exam_ids"],
  "prompt": "string",
  "tiempo_limite_segundos": "number"
}
\end{lstlisting}

\section{Retroalimentación}

\subsection{POST /api/generate-feedback}
\begin{itemize}
    \item \textbf{Descripción}: Genera retroalimentación personalizada
    \item \textbf{Autenticación}: Requerida
    \item \textbf{Body}:
\end{itemize}

\begin{lstlisting}
{
  "examen_id": "string"
}
\end{lstlisting}

\chapter{Base de Datos}

\section{Estructura de Tablas}

\subsection{Tabla: users}
\begin{lstlisting}
- id (uuid, primary key)
- email (varchar)
- created_at (timestamp)
- updated_at (timestamp)
\end{lstlisting}

\subsection{Tabla: exams}
\begin{lstlisting}
- id (uuid, primary key)
- user_id (uuid, foreign key)
- titulo (varchar)
- descripcion (text)
- preguntas (jsonb)
- dificultad (varchar)
- numero_preguntas (integer)
- tiempo_limite (integer)
- created_at (timestamp)
\end{lstlisting}

\subsection{Tabla: exam\_results}
\begin{lstlisting}
- id (uuid, primary key)
- exam_id (uuid, foreign key)
- user_id (uuid, foreign key)
- respuestas (jsonb)
- puntaje (integer)
- tiempo_empleado (integer)
- completed_at (timestamp)
\end{lstlisting}

\subsection{Tabla: feedback}
\begin{lstlisting}
- id (uuid, primary key)
- exam_id (uuid, foreign key)
- user_id (uuid, foreign key)
- feedback_data (jsonb)
- created_at (timestamp)
\end{lstlisting}

\chapter{Integración con IA}

\section{Google Gemini AI}

\subsection{Modelos Utilizados}
\begin{itemize}
    \item \textbf{gemini-2.0-flash}: Para retroalimentación y exámenes fáciles
    \item \textbf{gemini-2.5-pro-exp-03-25}: Para exámenes complejos y análisis avanzado
\end{itemize}

\subsection{Prompt Engineering}
El sistema utiliza instrucciones estructuradas para generar contenido consistente:

\begin{lstlisting}
const systemInstruction = `
Analiza el contenido y crea un examen con la siguiente estructura JSON:
{
  "dato": [
    {
      "id": 1,
      "pregunta": "Pregunta del examen",
      "opciones": ["opcion1", "opcion2", "opcion3", "opcion4"],
      "correcta": 0
    }
  ],
  "titulo": "Titulo del examen",
  "numero_preguntas": 5,
  "descripcion": "Descripcion del contenido",
  "dificultad": "easy|medium|hard|mixed"
}
`;
\end{lstlisting}

\subsection{Procesamiento de Archivos}
\begin{itemize}
    \item Soporte para múltiples formatos de documento
    \item Extracción de contenido para análisis
    \item Limpieza automática de archivos temporales
\end{itemize}

\chapter{Funcionalidades del Sistema}

\section{Gestión de Usuarios}
\begin{itemize}
    \item Registro y autenticación
    \item Perfiles de usuario
    \item Historial de exámenes
    \item Estadísticas personalizadas
\end{itemize}

\section{Generación de Exámenes}
\begin{itemize}
    \item A partir de documentos subidos
    \item Basado en prompts de texto
    \item Utilizando historial previo
    \item Configuración de dificultad y tiempo
\end{itemize}

\section{Realización de Exámenes}
\begin{itemize}
    \item Interfaz intuitiva de preguntas
    \item Timer configurable
    \item Navegación entre preguntas
    \item Guardado automático de progreso
\end{itemize}

\section{Retroalimentación}
\begin{itemize}
    \item Análisis detallado por pregunta
    \item Explicaciones de respuestas correctas
    \item Identificación de áreas de mejora
    \item Retroalimentación personalizada con IA
\end{itemize}

\section{Historial y Estadísticas}
\begin{itemize}
    \item Registro de exámenes completados
    \item Métricas de rendimiento
    \item Análisis de progreso
    \item Logros y metas
\end{itemize}

\chapter{Configuración de Entorno}

\section{Desarrollo}

\subsection{Frontend}
\begin{lstlisting}[language=bash]
cd frontend
npm run dev
\end{lstlisting}

\begin{itemize}
    \item Puerto: 5173 (Vite default)
    \item Hot reload activado
    \item DevTools habilitadas
\end{itemize}

\subsection{Backend}
\begin{lstlisting}[language=bash]
cd backend
node src/index.js
\end{lstlisting}

\begin{itemize}
    \item Puerto: 3001
    \item CORS configurado para desarrollo
\end{itemize}

\section{Producción}

\subsection{Build Frontend}
\begin{lstlisting}[language=bash]
cd frontend
npm run build
\end{lstlisting}

\subsection{Variables de Entorno de Producción}
\begin{itemize}
    \item Configurar URLs de produccion
    \item Usar HTTPS
    \item Configurar CORS específico
    \item Habilitar logs de produccion
\end{itemize}

\chapter{Despliegue}

\section{Frontend (Hostinger)}
\begin{lstlisting}[language=bash]
npm run build
# Subir carpeta dist/ al hosting
\end{lstlisting}

\section{Backend}
\begin{lstlisting}[language=bash]
# En servidor de produccion
git clone [repositorio]
cd backend
npm install --production
pm2 start src/index.js --name "exam-backend"
\end{lstlisting}

\section{Base de Datos}
\begin{itemize}
    \item Configurar Supabase en produccion
    \item Migrar esquemas si es necesario
    \item Configurar backups automáticos
\end{itemize}

\chapter{Mantenimiento}

\section{Logs}
\begin{itemize}
    \item Revisar logs de aplicacion regularmente
    \item Monitorear errores de IA
    \item Verificar uso de cuotas de API
\end{itemize}

\section{Base de Datos}
\begin{itemize}
    \item Backup regular de datos
    \item Limpieza de archivos temporales
    \item Optimización de consultas
\end{itemize}

\section{Actualizaciones}
\begin{itemize}
    \item Dependencias de Node.js
    \item Versiones de React
    \item APIs de Google Gemini
\end{itemize}

\chapter{Solución de Problemas}

\section{Errores Comunes}

\subsection{Error de API Key de Gemini}
\begin{lstlisting}
Error: Falta la variable de entorno GEMINI_API_KEY
\end{lstlisting}

\textbf{Solución}: Verificar que la variable esté configurada en \texttt{.env}

\subsection{Error de CORS}
\begin{lstlisting}
Access to fetch blocked by CORS policy
\end{lstlisting}

\textbf{Solución}: Verificar configuracion de CORS en backend

\subsection{Error de Autenticación Supabase}
\begin{lstlisting}
Invalid JWT token
\end{lstlisting}

\textbf{Solución}: Verificar configuracion de Supabase y tokens

\section{Monitoreo}

\subsection{Métricas Importantes}
\begin{itemize}
    \item Tiempo de respuesta de API
    \item Tasa de errores de generación
    \item Uso de cuota de Gemini AI
    \item Carga de base de datos
\end{itemize}

\subsection{Herramientas Recomendadas}
\begin{itemize}
    \item Logs de aplicacion
    \item Métricas de Supabase
    \item Google Cloud Console (para Gemini)
\end{itemize}

\chapter{Conclusión}

Este manual técnico proporciona una guía completa para la comprensión, instalación, configuracion y mantenimiento del Sistema de Generación de Exámenes con IA. Para actualizaciones y modificaciones, consultar el repositorio del proyecto y mantener actualizada la documentación.

\vspace{1cm}

\noindent\textbf{Versión del Manual}: 1.0\\
\textbf{Fecha}: Junio 2025\\
\textbf{Desarrollado por}: Equipo de Desarrollo

\end{document}